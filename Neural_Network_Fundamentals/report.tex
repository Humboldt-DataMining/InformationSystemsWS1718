\documentclass[a4paper, 11pt]{article}

\usepackage{graphicx}
\usepackage{amsmath}
% \usepackage{amsfonts}
\usepackage{amssymb}
\usepackage{authblk}

\usepackage{tikz}
\usetikzlibrary{arrows,backgrounds,snakes,patterns}
\usetikzlibrary{shapes,arrows,chains}
\usepackage{verbatim}
\usepackage{booktabs}

\usepackage[flushleft]{threeparttable}

% Added Packages ========
\usepackage[style=numeric-comp]{biblatex}
\addbibresource{dl-bibliography.bib} 

\usepackage[margin=1in]{geometry} % set margin to one inch
\usepackage{float}
\usepackage{blindtext}
% Code highlighting
\usepackage{listings} 
\usepackage{color}
% \usepackage{lmodern,textcomp}
\usepackage{tgbonum}

 
\definecolor{codegreen}{rgb}{0,0.6,0}
\definecolor{codegray}{rgb}{0.5,0.5,0.5}
\definecolor{codepurple}{rgb}{0.58,0,0.82}
\definecolor{backcolour}{rgb}{0.95,0.95,0.92}

% \lstset{basicstyle=\footnotesize\ttfamily,breaklines=true}


\lstdefinestyle{mystyle}{
    backgroundcolor=\color{backcolour},   
    commentstyle=\color{codegreen},
    keywordstyle=\color{magenta},
    numberstyle=\tiny\color{codegray},
    stringstyle=\color{codepurple},
    basicstyle=\footnotesize\ttfamily,
    breakatwhitespace=false,         
    breaklines=true,                 
    captionpos=b,                    
    keepspaces=true,                 
    numbers=left,                    
    numbersep=5pt,                  
    showspaces=false,                
    showstringspaces=false,
    showtabs=false,                  
    tabsize=2
}
 
\lstset{style=mystyle}

% 

\begin{document}

\title{Fundamentals of Neural Networks}

\author[1]{Denis Augusto Pinto Maciel}
\author[1]{Roman Pro}
\author[1]{Mahdi Bayat}

\affil[1]{Humboldt University of Berlin, Berlin, Germany}

\maketitle

\begin{abstract}
This article is a guide to help readers build a neural network from the very basics. It starts with an introduction to the concept of a neural networks concept and its early development. A step-by-step coding tutorial follows, through which relevant concepts are illustrated. Later in the post, there is also an introduction on how to build neural networks in Keras. Finally, the reader will find instructions on how to deploy the model via an API to make it accessible to anyone interested on it.
\end{abstract}

\section{Introduction}
\label{sec:intro}
As the name tells, the idea of neural networks is inspired by how neurons work in the human brain. It is, however, crucial for the readers to know that despite the original motivation of neural networks, the NN models being used today have little resemblance to what a human brain does (Warner and Misra, 1996).  In its basic form, neural networks are composed of nodes interconnected to each other in several layers. The basic form of a NN would include an input, a hidden and an output layer. The number of nodes and layers can add to the complexity and efficiency of neural networks.  

The McCulloch-Pitts model of neuron in 1943 was one of the earliest simplified version of neural networks. It consisted of a simple neuron which received a weighted sum of inputs and output either zero if the sum was smaller than a threshold or one when it was greater than the threshold. This idea is called firing and is an interesting analogy to what an actual neuron does. Later on, in the early 1960s, Rosenblatt introduced the simple perceptron model. This was a developed version of the McCulloch-Pitts with an input and output layer. However, the linear separablilty limitation (Minsky and Papert ,1969) of simple perceptron took away the research interest in neural networks for a while. In the early 1980s, the Hopfield model of content-addressable memory, however, motivated researchers in the area again and later on with the introduction of backpropagation learning algorithm, interest in neural networks research soared. Nowadays, neural nets are used in a variety of applications to tackle problems such as classification, speech and image recognition, control systems and predictions.

\cite{chollet2018deep}

\section{Deep Learning with Keras}
\label{sec:keras}
We are now going to reimplement the previous neural network with the Keras framework. Keras is an open source neural network library written in Python. It has the advantage of abstracting most of the boiler-plate code one needs to write when implementing a neural net only with a linear algebra library. Thus, it is suitable to fast prototyping and experimentation.

It's important you make sure you have the required libraries installed for it to work. We will make use of three main libraries and their dependencies, which will be automatically installed.

If you are using Anconda's Python distribution, we advise you to run \lstinline{conda install keras pandas numpy} on the terminal. 
Otherwise, using the \lstinline{pip} package manager should also do the trick.
Run \lstinline{pip install keras pandas numpy} on the terminal.

The version of Python that will be used throughout this notebook is Python 3.6.4 from Anaconda's distribution. You can check your version of Python by executing the cell below.

\begin{lstlisting}[language=Python]
    import numpy as np
    from keras.models import Sequential
    from keras.layers import Dense
    from keras.utils import np_utils
    
    import pandas as pd
\end{lstlisting}

\subsection{Load the data}

To load the data, we will use the very handy pandas' \lstinline{read_csv} function. It frees us from the burden of parsing the text file.

\begin{lstlisting}[language=Python]
    names = ["y"] + list(range(1,785))
    df = pd.read_csv("data/mnist_train.csv", 
                    names=names)

    df_test = pd.read_csv("data/mnist_test.csv", 
                        names=names)

    df_test.head()
\end{lstlisting}

Next we separate labels from features in both train and test set and transform them from dataframes to numpy arrays, which are better suited for modeling.

\begin{lstlisting}[language=Python]
y_train = df['y'].values
X_train = df.iloc[:, 1:].values/255*0.99+0.01

y_test = df_test['y'].values
X_test = df_test.iloc[:, 1:].values/255*0.99+0.01

[y_train, y_test, X_train, X_test]
\end{lstlisting}

We now check if the shape of the arrays correspond to the expected. In fact, the shape is correct. We have 60 thousand observations in the train set and 10 thousand in the test set.

\begin{lstlisting}[language=Python]
[y_train.shape, X_train.shape, y_test.shape, X_test.shape]
\end{lstlisting}

Before defining the model, one extra step is necessary: transform the labels so they are one-hot encoded. One-hot encoding a vector means transforming it into a matrix of ones and zeroes only with as many columns as the number of different values in the vector. In the specific case, the label vector becomes a ten-column array, each column representing one digit. If the label of the observation is 2, it will have zeroes in columns expect in the third column, which will have a one. The number of rows remains the same. with the same number of rows as before.

\begin{lstlisting}[language=Python]
y_train = np_utils.to_categorical(y_train)
y_test = np_utils.to_categorical(y_test)

[y_test.shape, y_train.shape]
\end{lstlisting}

\subsection{Model}

\subsubsection{Define the model}

We finally come to the most important part. We will accomplish the task of building the neural network with only eight lines of code.

The model in question consists of one input, one hidden and one output layer. The activation function of the hidden layer is a ReLU. And we use as the optimizer Stochastic Gradient Descent. 

Keras makes it very simple to add new layers. One needs only to call the `add` method on the model and pass the layer with its specifications. As you can see, the number of inputs needs to be specified only in the first layer. Keras infers the input number of a layer by looking at the number of outputs of its predecessor.

For this neural network, we will only use dense layers, which are layers with all nodes fully connected to each other. Keras, however, allows you to arbitrarily build your neural networks by providing different types of layers, such as convolutional and pooling layers.

\begin{lstlisting}[language=Python]
    def baseline_model(num_hidden_n, num_pixels, num_classes, optimizer):
        model = Sequential()
        model.add(Dense(num_hidden_n, input_dim=num_pixels, kernel_initializer='normal', activation='relu'))
        model.add(Dense(num_classes, kernel_initializer='normal', activation='softmax'))
        
        # Compile model
        model.compile(loss='categorical_crossentropy', 
                    optimizer=optimizer,
                    metrics=['accuracy'])
        return model
\end{lstlisting}

\subsubsection{Instantiate the model}

Having defined the structure of the model, we can now instantiate a concrete version of it by picking the relevant parameters and calling the function that returns the model object.

Here we have chosen the hidden layers to have 90 nodes, while input and output layers have 784 and 10 nodes respectively. 

\begin{lstlisting}[language=Python]
    num_pixels, num_hidden_n, num_classes = 784, 90, 10
    optimizer = 'sgd'
    
    model = baseline_model(num_hidden_n, num_pixels, num_classes, optimizer)
\end{lstlisting}

\subsubsection{Train and evaluate the model}

With the model instantiated, we can finally call the fit method on it using the data set we prepared before.

After training the model we evaluate its performance by looking at its accuracy.

\begin{lstlisting}[language=Python]
    model.fit(X_train,
        y_train, 
        epochs=5,
        batch_size=200,
        verbose=2)
    scores = model.evaluate(X_test, y_test, verbose=0)
    print("Baseline Error: %.2f%%" % (100-scores[1]*100))
\end{lstlisting}

\begin{lstlisting}
    Epoch 1/5
    - 2s - loss: 1.9537 - acc: 0.5054
   Epoch 2/5
    - 2s - loss: 1.1453 - acc: 0.7680
   Epoch 3/5
    - 2s - loss: 0.7509 - acc: 0.8296
   Epoch 4/5
    - 2s - loss: 0.5942 - acc: 0.8552
   Epoch 5/5
    - 2s - loss: 0.5129 - acc: 0.8706
   Baseline Error: 11.96%
\end{lstlisting}

The error rate does not seem very good. Maybe we could try a different optimizer. We will instantiate and fit the model again with the RMSprop optimization algorithm. By using Keras, the only thing you need to do is to pass a different argument to the model.

\begin{lstlisting}[language=Python]
    model = baseline_model(num_hidden_n,
        num_pixels,
        num_classes,
        optimizer = "rmsprop")

    model.fit(X_train,
    y_train, 
    epochs=5,
    batch_size=200,
    verbose=2)
    scores = model.evaluate(X_test, y_test, verbose=0)
    print("Baseline Error: %.2f%%" % (100-scores[1]*100))
\end{lstlisting}

\begin{lstlisting}
    Epoch 1/5
    - 2s - loss: 0.4963 - acc: 0.8723
   Epoch 2/5
    - 2s - loss: 0.2401 - acc: 0.9306
   Epoch 3/5
    - 2s - loss: 0.1828 - acc: 0.9477
   Epoch 4/5
    - 2s - loss: 0.1466 - acc: 0.9581
   Epoch 5/5
    - 2s - loss: 0.1223 - acc: 0.9640
   Baseline Error: 3.54%   
\end{lstlisting}

\subsubsection{Save the model}

Once trained, you might want to use the model in the future. You can do so by saving it to a file for later use. Keras comes equipped with the `save` method, which allows you to easily save your trained model to the disk.

We are going to save the model into a file called `model.h5` and delete it from memory.

\begin{lstlisting}[language=Python]
    model.save("model.h5")
    del model    
\end{lstlisting}

Then we load the model from the file we just created and evaluate it again to make sure that during the saving process, the model hasn't been corrupted. The base line error is the same: the model has been successfully saved and can be shared with third-parties. 

\begin{lstlisting}[language=Python]
    from keras.models import load_model

    model2 = load_model("model.h5")
    
    scores2 = model2.evaluate(X_test, y_test, verbose=0)
    print("Baseline Error: %.2f%%" % (100-scores2[1]*100))    
\end{lstlisting}


% BibTeX users please use one of
%\bibliographystyle{spbasic}    % basic style, author-year citations
%\bibliographystyle{spmpsci}     % mathematics and physical sciences
%\bibliographystyle{spphys}     % APS-like style for physics
% \bibliographystyle{plain}
% \bibliography{references}       % name your BibTeX data base

\printbibliography
\end{document}
% end of file template.tex


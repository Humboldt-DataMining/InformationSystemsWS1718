As the name tells, the idea of neural networks is inspired by how neurons work in the human brain. It is, however, crucial for the readers to know that despite the original motivation of neural networks, the NN models being used today have little resemblance to what a human brain does (Warner and Misra, 1996).  In its basic form, neural networks are composed of nodes interconnected to each other in several layers. The basic form of a NN would include an input, a hidden and an output layer. The number of nodes and layers can add to the complexity and efficiency of neural networks.  

The McCulloch-Pitts model of neuron in 1943 was one of the earliest simplified version of neural networks. It consisted of a simple neuron which received a weighted sum of inputs and output either zero if the sum was smaller than a threshold or one when it was greater than the threshold. This idea is called firing and is an interesting analogy to what an actual neuron does. Later on, in the early 1960s, Rosenblatt introduced the simple perceptron model. This was a developed version of the McCulloch-Pitts with an input and output layer. However, the linear separablilty limitation (Minsky and Papert ,1969) of simple perceptron took away the research interest in neural networks for a while. In the early 1980s, the Hopfield model of content-addressable memory, however, motivated researchers in the area again and later on with the introduction of backpropagation learning algorithm, interest in neural networks research soared. Nowadays, neural nets are used in a variety of applications to tackle problems such as classification, speech and image recognition, control systems and predictions.

We are now going to reimplement the previous neural network with the Keras framework. Keras is an open source neural network library written in Python. It has the advantage of abstracting most of the boiler-plate code one needs to write when implementing a neural net only with a linear algebra library. Thus, it is suitable to fast prototyping and experimentation.

It's important you make sure you have the required libraries installed for it to work. We will make use of three main libraries and their dependencies, which will be automatically installed.

If you are using Anconda's Python distribution, we advise you to run \lstinline{conda install keras pandas numpy} on the terminal. 
Otherwise, using the \lstinline{pip} package manager should also do the trick.
Run \lstinline{pip install keras pandas numpy} on the terminal.

The version of Python that will be used throughout this notebook is Python 3.6.4 from Anaconda's distribution. You can check your version of Python by executing the cell below.

\begin{lstlisting}[language=Python]
    import numpy as np
    from keras.models import Sequential
    from keras.layers import Dense
    from keras.utils import np_utils
    
    import pandas as pd
\end{lstlisting}

\subsection{Load the data}

To load the data, we will use the very handy pandas' \lstinline{read_csv} function. It frees us from the burden of parsing the text file.

\begin{lstlisting}[language=Python]
    names = ["y"] + list(range(1,785))
    df = pd.read_csv("data/mnist_train.csv", 
                    names=names)

    df_test = pd.read_csv("data/mnist_test.csv", 
                        names=names)

    df_test.head()
\end{lstlisting}

Next we separate labels from features in both train and test set and transform them from dataframes to numpy arrays, which are better suited for modeling.

\begin{lstlisting}[language=Python]
y_train = df['y'].values
X_train = df.iloc[:, 1:].values/255*0.99+0.01

y_test = df_test['y'].values
X_test = df_test.iloc[:, 1:].values/255*0.99+0.01

[y_train, y_test, X_train, X_test]
\end{lstlisting}

We now check if the shape of the arrays correspond to the expected. In fact, the shape is correct. We have 60 thousand observations in the train set and 10 thousand in the test set.

\begin{lstlisting}[language=Python]
[y_train.shape, X_train.shape, y_test.shape, X_test.shape]
\end{lstlisting}

Before defining the model, one extra step is necessary: transform the labels so they are one-hot encoded. One-hot encoding a vector means transforming it into a matrix of ones and zeroes only with as many columns as the number of different values in the vector. In the specific case, the label vector becomes a ten-column array, each column representing one digit. If the label of the observation is 2, it will have zeroes in columns expect in the third column, which will have a one. The number of rows remains the same. with the same number of rows as before.

\begin{lstlisting}[language=Python]
y_train = np_utils.to_categorical(y_train)
y_test = np_utils.to_categorical(y_test)

[y_test.shape, y_train.shape]
\end{lstlisting}

\subsection{Model}

We finally come to the most important part. We will accomplish the task of building the neural network with only eight lines of code. But before we dive into that, a few words about the activation function and the optimizer are due.

\subsubsection{Activation Functions and Gradient Descent}

The activation functions are compared to each other in terms of their efficient approximation (Schnitger & DasGupta, 1993). There are several types of them but the common ones are the sigmoid, tangent hyperbolic and the ReLU functions. The following is a short explanation on the advantages and disadvantages of each of these activation functions. 

\begin{itemize}
    \item Sigmoid function: This function squashes the input value into a value in the range of [0,1]. It is powerful to compute bolean functions and the smoothness of sigmoid allows for better approximation \cite{dasgupta1993power}. However, there are two major drawbacks. First, the sigmoid outputs 0 for high negative values and 1 for high positive values and this result in no signal flows from the nodes because the gradient at these two tails are 0. Second, its output is not centered at 0. This would ultimately result in either all positive gradients on weights or all negative during the backpropogation.
    \item Tangent hyperbolic: Unlike to sigmoid, this function’s output is zero cantered and outputs a value in the range of [-1,1]. Although it is more preferred over sigmoid but it suffers from the same problem because it kills the gradient descent at high or low values.
    \item ReLU: This function is zero whenever input is negative and when the input is positive, its behavior is linear with slope 1. According to \cite{glorot2011domain}, the ReLU activation functions outperform sigmoidal functions in Deep Neural Networks on text classification and image recognition tasks. ReLU also excels the sigmoid and tangent hyperbolic in accelerating the stochastic gradient descent convergence. However, the ReLU is not prone to the training process and the nodes can die. This means that if a large gradient runs through the nodes, it may update the weights in a way that the node will never be active for any inputs again and this will result in gradient of zero from that node onward.
\end{itemize}

As mentioned before, gradient descent is one the most popular algorithms to optimize the neural networks. There are many variations of it. The most popular of them are Mini-batch Gradient Descent and the Stochastic Gradient Descent. 

\begin{itemize}
    \item The \textbf{Mini-batch Gradient Descent (BGD)} is used when there is a large scale training data set where the computation of the cost function over the whole training data set is computationally expensive. Therefore, a batch of the training data set, say 250 observations, is randomly selected to compute the gradient with the assumption that the observations in the training dataset are correlated. 
    \item If this procedure is done only one observation instead of 250, but with 250 iterations, then this procedure is called \textbf{Stochastic Gradient Descent (SGD)}. This approach is repeatedly sampling a training example, which is used to evaluate the gradient. Sometimes it is considered less efficient as the loss function calculated 250 times based on a single sampled training example can be noisy.
\end{itemize}

You can learn more about different types of activation functions and optimizers using the official documentation of Keras:

\begin{itemize}
    \item https://keras.io/optimizers/
    \item https://keras.io/activations/
\end{itemize}

A very insightful overview of optimization mechanisms is done by \cite{ruder2016overview}.

\subsubsection{Define the model}

The model in question consists of one input, one hidden and one output layer. The activation function of the hidden layer is a ReLU. And we use as the optimizer Stochastic Gradient Descent. 

Once the activation function and optimizers are selected it is time to determine the structure of the NN. Keras makes it very simple to add new layers. One needs only to call the add method on the model and pass the layer with its specifications. As you can see, the number of inputs needs to be specified only in the first layer. Keras infers the input number of a layer by looking at the number of outputs of its predecessor.

For this neural network, we will only use dense layers, which are layers with all nodes fully connected to each other. Keras, however, allows you to arbitrarily build your neural networks by providing different types of layers, such as convolutional and pooling layers.

Note that we will isolate the model definition into its own unity, a function. We might experiment with different specifications for the optimizer and by wrapping the model in a function avoids duplicating code.

\begin{lstlisting}[language=Python]
    def baseline_model(num_hidden_n, num_pixels, num_classes, optimizer):
        model = Sequential()
        model.add(Dense(num_hidden_n, input_dim=num_pixels, kernel_initializer='normal', activation='relu'))
        model.add(Dense(num_classes, kernel_initializer='normal', activation='softmax'))
        
        # Compile model
        model.compile(loss='categorical_crossentropy', 
                    optimizer=optimizer,
                    metrics=['accuracy'])
        return model
\end{lstlisting}

\subsubsection{Instantiate the model}

Having defined the structure of the model, we can now instantiate a concrete version of it by picking the relevant parameters and calling the function that returns the model object.

Here we have chosen the hidden layers to have 90 nodes, while input and output layers have 784 and 10 nodes respectively. 

\begin{lstlisting}[language=Python]
    num_pixels, num_hidden_n, num_classes = 784, 90, 10
    optimizer = 'sgd'
    
    model = baseline_model(num_hidden_n, num_pixels, num_classes, optimizer)
\end{lstlisting}

\subsubsection{Train and evaluate the model}

With the model instantiated, we can finally call the fit method on it using the data set we prepared before.

After training the model we evaluate its performance by looking at its accuracy.

\begin{lstlisting}[language=Python]
    model.fit(X_train,
        y_train, 
        epochs=5,
        batch_size=200,
        verbose=2)
    scores = model.evaluate(X_test, y_test, verbose=0)
    print("Baseline Error: %.2f%%" % (100-scores[1]*100))
\end{lstlisting}

\begin{lstlisting}
    Epoch 1/5
    - 2s - loss: 1.9537 - acc: 0.5054
   Epoch 2/5
    - 2s - loss: 1.1453 - acc: 0.7680
   Epoch 3/5
    - 2s - loss: 0.7509 - acc: 0.8296
   Epoch 4/5
    - 2s - loss: 0.5942 - acc: 0.8552
   Epoch 5/5
    - 2s - loss: 0.5129 - acc: 0.8706
   Baseline Error: 11.96%
\end{lstlisting}

The error rate does not seem very good. Maybe we could try a different optimizer. We will instantiate and fit the model again with the RMSprop optimization algorithm. By using Keras, the only thing you need to do is to pass a different argument to the model.

\begin{lstlisting}[language=Python]
    model = baseline_model(num_hidden_n,
        num_pixels,
        num_classes,
        optimizer = "rmsprop")

    model.fit(X_train,
    y_train, 
    epochs=5,
    batch_size=200,
    verbose=2)
    scores = model.evaluate(X_test, y_test, verbose=0)
    print("Baseline Error: %.2f%%" % (100-scores[1]*100))
\end{lstlisting}

\begin{lstlisting}
    Epoch 1/5
    - 2s - loss: 0.4963 - acc: 0.8723
   Epoch 2/5
    - 2s - loss: 0.2401 - acc: 0.9306
   Epoch 3/5
    - 2s - loss: 0.1828 - acc: 0.9477
   Epoch 4/5
    - 2s - loss: 0.1466 - acc: 0.9581
   Epoch 5/5
    - 2s - loss: 0.1223 - acc: 0.9640
   Baseline Error: 3.54%   
\end{lstlisting}

\subsubsection{Save the model}

Once trained, you might want to use the model in the future. You can do so by saving it to a file for later use. Keras comes equipped with the `save` method, which allows you to easily save your trained model to the disk.

We are going to save the model into a file called `model.h5` and delete it from memory.

\begin{lstlisting}[language=Python]
    model.save("model.h5")
    del model    
\end{lstlisting}

Then we load the model from the file we just created and evaluate it again to make sure that during the saving process, the model hasn't been corrupted. The base line error is the same: the model has been successfully saved and can be shared with third-parties. 

\begin{lstlisting}[language=Python]
    from keras.models import load_model

    model2 = load_model("model.h5")
    
    scores2 = model2.evaluate(X_test, y_test, verbose=0)
    print("Baseline Error: %.2f%%" % (100-scores2[1]*100))    
\end{lstlisting}
